% Time-stamp: <2025-05-28 17:17:04 amano>

\documentclass[11pt]{article}

\pagestyle{empty}
\usepackage{color}
\parindent=0pt
\parskip=5pt


\title{First-principles study of THz dielectric properties of molecular liquids with a machine learning model for dipole moments}
\author{Tomohito Amano, Tamio Yamazaki, Shinji Tsuneyuki}

\begin{document}
\maketitle


% -------------------------------------------
% General info and title
% -------------------------------------------

SORTING CATEGORY: B5 | Emerging artificial intelligence methods for computational materials discovery

% -------------------------------------------
ABSTRACT:
% -------------------------------------------

Understanding the dielectric response of materials in the GHz to THz region is crucial for both fundamental science and technological applications. We develop a versatile machine learning framework of dipole moments to predict dielectric functions from first principles with high accuracy\footnote{https://github.com/ToAmano/MLWC}. We assigned Wannier functions to chemical bonds between atoms and used deep neural networks to predict the position of the Wannier function for each bond, which is applicable to complex materials. We demonstrate that our method accurately reproduces the dielectric spectra of liquid methanol and ethanol, providing new insights into the physical origin of THz absorption peaks\footnote{T. Amano, T. Yamazaki, and S. Tsuneyuki, Chemical bond based machine learning model for dipole moment: Application to dielectric properties of liquid methanol and ethanol, Phys. Rev. B 110, 165159 (2024).}. Furthermore, we validate the transferability of our model across different systems by successfully applying it to liquid propylene glycol and polypropylene glycol\footnote{T. Amano, T. Yamazaki, N. Matsumura, Y. Yoshimoto, and S. Tsuneyuki, Transferability of the chemical-bond-based machine learning model for dipole moment: The GHz to THz dielectric properties of liquid propylene glycol and polypropylene glycol, Phys. Rev. B 111, 165149 (2025).}. Our results open a pathway toward efficient and accurate calculation of dielectric properties in complex molecular liquids.


\end{document}


